\subsection{Functional Requirements}
\begin{flushleft}

\subsubsection{SubmitPost}
\begin{flushleft}
\textbf{Test Status:} \emph{Partial Pass}. \\

\textbf{Details:}
The submit post use-case was divided into seperate smaller functions. One function \emph{create(mUser, mParent, mPostType, mHeading, mContent, mMimeType)} is used to initialize the process of creating a new thread as well as its embedded post object. This function calls \emph{createNewThread(mUser, mParent, mLevel, mPostType, mHeading, mContent, mMimeType, mSubject)} which parses the current thread and post object to JSON strings and then persists them in a remote database.\\

The problem however is that when child threads are created using the \emph{createThread(...)} function call, the child thread is never persisted to the remote datase.\\

The functional requirements also state clearly that a thread (with its related post) should have been allocated a certain space, yet this functionality is never provided.

\end{flushleft}
\subsubsection{MarkPostAsRead}
\begin{flushleft}
\textbf{Test Status:} \emph{Complete Failure}. \\

\textbf{Details:}
The teams approach to the reading events are not that well planned and clearly thought through. It simply sets a flag for the thread and persists that flag to the database. This however will then not be user based, as everyone who then accesses that specific thread from the database will have ``read'' it.\\
This function is hence not working as required and thus fails completely. It is however a nice-to-have function and does not break or affect any other part of the thread module.

\end{flushleft}

\subsubsection{Overview of Tests Results for Threads B- Functions MoveThread and HideThread}
\begin{flushleft}
\textbf{Tests log:}
The BuzzSpace software was tested on the nodejs test platform using the nodeunit environment located in the respective github repository for group ThreadB, from the 2015/04/20 to the 2015/04/24. The tests of the test phase (ref. software test plan) where executed.


\textbf{Functions Tested:}
\emph{moveThread(...)} 
\emph{hideThread(...)} 
\emph{unHideThread(...)} 

\textbf{Rationale for decision:}
After executing a test, the decision is defined according to the following rules:
\begin{itemize}
  	\item OK: The test result is compliant to the expected result.
  	\item 	NOK: The test result differs from the expected result.
   	\item Partial OK: There is partially compliant to the expected result. 
    	\item NOT RUN: The default state of a test sheet cannot be executed.
  	\item NOT COMPLETED: The test sheet is set to "Not Completed" state when at least one step of the test is set "Not Run" state. \ldots
\end{itemize}

\textbf{Overall assessment of tests:}
Overall assessment of tests:
\begin{itemize}
	\item All tests did not provide an interface.
	\item 	All tests passed but with the use of different parameters were not supposed to pass.
	\item The tests respond to all kinds of input but they respond incorrectly.
	\item Tests achieve the desired input in the database. 
 \end{itemize}

\textbf{Techniques used to test:}
\begin{enumerate}
  	\item All tests did not provide an interface.
  	\item Code average- a check to see if all statement executed only once
   	\item Fault injection – I placed invalid expectations of the result and an error was not thrown.
    	\item Software 
    	\item Excercising inputs and checking the output for a differ in behavior
 \end{enumerate}

\textbf{Statistics about tests:}
\begin{itemize}
  	\item  0 percent of tests OK
  	\item  0 percent of tests NOK
   	\item 100 percent of tests POK
    	\item 	0 percent of tests NR
        	\item 	0 percent of tests NC
 \end{itemize}

\textbf{Test cases:}
\begin{itemize}
  	\item Test cases planned = 4 
  	\item  Test cases executed = 4
   	\item Test cases passed = 0
    	\item Test cases failed = 4
 \end{itemize}

\textbf{Give also statistics about bugs and enhancements:}
\begin{itemize}
  	\item 	Total number = 4
  	\item  	Number of Critical = 4
   	\item 	Number of Major = 0
    	\item Number of minor = 0
      	\item Number of enhancements = 0
 \end{itemize}
 \end{flushleft}
 
\begin{flushleft}
\textbf{Conclusion:}
The bugs specified were Mongoose related. (Error: Cast Error:Cast to ObjectId failed for value [object Object] at path\_id)\\
In HideThread we changed the parameter in test.equal to a different expected result and it still passed so we don’t think the function actually checks if the value field is hidden or not.\\
In MoveThread there are no error checks or bounds checks as we inserted an overbound value and it still passed the test.
\end{flushleft}



%CONTENT

\end{flushleft}
