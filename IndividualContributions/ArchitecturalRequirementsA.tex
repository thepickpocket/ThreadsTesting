\subsection{Architectural Requirements}
\begin{flushleft}

%CONTENT
The Threads module has the role of managing the threads of a buzz space in the Buzz system. This module has been made to be non-blocking so that multiple calls for services to this module will result in independent responses. This module is apart of the core modules used in the Buzz system.

The Threads module is implemented in Node(as in Node.js) in conjunction with the Node Package Manager(NPM). Node is by nature a non-blocking run-time environment using Java-Script. The installation of various packages from NPM allow for the different architectural requirements to be fulfilled. 

\subsubsection{Scope of responsibility}
The Threads module encompasses a persistence infrastructure and a reporting infrastructure. 

The persistent infrastructure that is used by this module is MongoDB. MongoDB is an open-source document database. The MongoDB database is accessed by

The reporting infrastructure is in the form functions or services that return/report the status of a request whether it be successful or not.

\subsubsection{Quality requirements}
The quality requirements addressed by this module are the following:
\begin{itemize}
\item Performance requirements
\item Maintainability
\item Testability
\item Usability
\end{itemize}
The module is maintainable as the functional requirements for Threads are implemented with separation of concerns in mind. 
The module is very testable and has unit tests performed on each of the exported functions. Because of the require function provided in node, it is easily usable.
\b
The ones that are not addressed are:
\begin{itemize}
\item Scalability
\end{itemize}
The system module is not very scalable mainly because of the database management system, MongoDB. MongoDB is a light-weight database system and would not be suitable on a large scale. Otherwise the module would not have scalability issues.

\subsubsection{Integration and access channels}
Those who have access to this module are those that make use of the Buzz system. For example, a registered user, registered to a Buzz space, will have CRUD operations available to them.

\subsubsection{Architectural constraints}
The architectural constraints on this module would be the use of the Node run-time environment. There is no restriction on the database or the node modules used.

\subsubsection{Technologies}
\begin{itemize}
\item Node
\item NPM
\item Mongoose
\item MongoDB
\end{itemize}

\end{flushleft}
