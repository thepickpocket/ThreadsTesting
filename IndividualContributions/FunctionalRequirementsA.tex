\subsection{Functional Requirements.}
\begin{flushleft}

\subsubsection{MoveThread}
\begin{flushleft}
	\textbf{Test Status:} \emph{Partial Pass}. \\

	\textbf{Details:}
	The main objective of the MoveThread function is to detach a sub-tree of thread nodes from one thread and add it to another thread.\\
	The services contract MoveThreadRequest requires only one pre-condition, that the user is a administrator.
	The services contract was successfully implemented. This function only managed to pass partially. The MoveThread function depends on the functionality of the HideThread and UnHideThread functions. If these functions was not required during testing the thread could successfully be moved to another thread.\\
	One of the main requirements for all functions is that the function can be used on threads in the database. This is not true for this function as it only works locally. There is no functionality to retrieve the required threads from the database.

\end{flushleft}

\subsubsection{HideThread}
\begin{flushleft}
	\textbf{Test Status:} \emph{Complete Failure}. \\

	\textbf{Details:}
	The main objective of the HideThread function is that selected thread nodes and all	its descendant nodes will be marked as hidden and user interfaces are meant not to render them.\\
	The HideThread function was partially implemented. The services contract HideThreadRequest requires only one pre-condition, that the user is a administrator. The services contract was successfully implemented. The functionality of the HideThread function itself was not implemented. The priority of the HideThread function is stated as important thus the system will still function as it is not a critical function within the Thread module.
  
\end{flushleft}
%CONTENT

\end{flushleft}
